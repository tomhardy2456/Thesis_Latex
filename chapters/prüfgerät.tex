\chapter{Das Prüfgerät}
Dabei werden die Elastomerproben in Form von Stäbchen horizontal in ein Prüfgerät eingespannt, wobei die eine Seite fest und die andere Seite beweglich ist. \ref{fig:part01:prüfgerät}. Während des Testdurchlaufs wird  mittels kontinuierlicher Aufnahme von Fotos der Zeitpunkt protokolliert, an dem eine Probe reißt, und diese Information wird später im Analyseprozess verwendet. 

\begin{figure}[htbp]
 \centering
 \includegraphics[width=1\textwidth]{gfx/prüfgerät.png}
 \caption{Draufsichten auf das Prüfgerät mit 8 Elastomerproben. Eine Einspannungsseite ist feststehend, die andere wird horizontal zyklisch bewegt.}
 \label{fig:part01:prüfgerät}
\end{figure}

\begin{figure}[htbp]
 \centering
 \includegraphics[width=1\textwidth]{gfx/image_5.png}
 \caption{Prüfgerät zur zyklischen Elastomerprüfung unter Temperatureinfluss. Quelle \cite{zyklischePrüfung}}
 \label{fig:part01:prüfgerät_2}
\end{figure}