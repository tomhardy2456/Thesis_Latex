\chapter{Einleitung}
Elastomere kommen in der Autoindustrie aufgrund ihrer elastischen Eigenschaften und ihrer Widerstandsfähigkeit gegenüber verschiedenen Umweltbedingungen in verschiedenen Anwendungen zum Einsatz. Dazu zählen unter anderem Reifen, Lenkungskomponenten, Dichtungen, Lager, und weitere Komponenten. Im Laufe der Zeit unterliegen Elastomerkomponenten in Fahrzeugen verschiedenen Alterungseinflüssen, wie thermomechanischer Belastung, chemischen Substanzen, Feuchtigkeit und Strahlung. Daher ist es von entscheidender Bedeutung für Automobilhersteller, die Mechanismen der Alterung und des Versagens von Elastomeren zu verstehen und diese dementsprechend aus besonders widerstandsfähigem Material herzustellen.

Zur Ableitung des Alterungsverhaltens von Elastomeren wird die Schwingfestigkeit durch Ermüdungstest (Wöhler-Versuche) mithilfe einer dynamischen Zugprüfmaschine ermittelt \cite{elastomeralterung}. Die daraus gewonnenen Informationen dienen als Grundlage für die numerische Analyse des Übertragungsverhaltens und die Schätzung der Lebensdauer von Elastomerbauteilen.

Ein bestehendes Problem liegt in der Dauer solches Ermüdungstests, die oft mehrere Tage bis Wochen in Anspruch nehmen. Dies erfordert einen erheblichen Aufwand, da regelmäßige manuelle Überprüfungen und die manuelle Analyse von Tausenden von Bildern am Ende erforderlich sind. Eine Implementierung eines KI-Systems erscheint daher sinnvoll, um den zeitaufwändigen Prozess der manuellen Erkennung von gerissenen Proben zu automatisieren. Die Aufgabe eines solchen KI-Systems besteht darin, die Zugprüfmaschine während des gesamten Versuchs zu überwachen und den Zeitpunkt des Versagens einer Elastomerprobe zu protokollieren.

Diese Abschlussarbeit schlägt einen Ansatz zur Entwicklung eines KI-Systems für die automatische Fehlererkennung in zyklischen Prüfungen (Ermüdungstests) von Elastomeren vor. Dieses Problem wird als "real-time Multi-Object-Tracking (\acs{MOT})"  klassifiziert, weil das System mehrere gerissenen Elatomerproben in einem Videostream in Echtzeit erkennen, tracken und identifizieren muss. Für das \acs{MOT} Problem wird das Tracking-by-Detection Paradigma sehr häufig verwendet \cite{yu2016poi} \cite{aharon2022botsort} \cite{zhang2022bytetrack} \cite{Chen_2018} \cite{wojke2017simple}. In dieser Arbeit wird ... \textbf{(Hier werde ich die Struktur meiner Arbeit zusammenfassen)}

\section{Forschungsfragen}
